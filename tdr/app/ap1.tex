\begin{refsection}
\chapter{Desenvolupament del treball}

	En el primer bloc de la memòria s'ha explicat la part més teòrica del treball que consta de 3 apartats: \nameref{chap:AI}, \nameref{chap:ML} i \nameref{chap:ANN}. Aquesta part del treball inclou la recerca d'informació i, quan sigui possible, alguns exemples pràctics per complementar la teoria. En la recerca bibliogràfica s'han utilitzat diverses fonts:

	\begin{itemize}

		\item Com és natural, la font d'informació que més he utilitzat, la que predominarà en la part teòrica del treball i la que m’ha clarificat més conceptes ha estat la recerca amb els buscadors d'Internet. Per aquesta font he aconseguit molts treballs relacionats amb la intel·ligència artificial i l'aprenentatge automàtic. Entre els diferents treballs, destaquen els treballs de fi de grau (TFG) i els projectes de final de carrera (PFC) en l'àmbit de les Tecnologies de la Informació i les Comunicacions, sobretot l'Enginyeria Informàtica. Principalment, tota la informació d'aquesta font és en idioma anglès i extreta dels repositoris o dipòsits digitals de les universitats, com ara la UPF\supercite{eRepositoriUPF}, la UAB\supercite{DDDUAB} o la UPC.\supercite{UPCommons}

		\item A parts dels repositoris, he fet servir bases de dades d'autors i d'articles publicats a revistes científiques internacionals, amb l'ajuda del motor de cerca Google Acadèmic. M'agradaria destacar algunes pàgines web que m'han resultat molt útils:

		      \begin{itemize}

			      \item \textbf{Google Acadèmic} és un motor de cerca de Google que indexa el text complet o les metadades de literatura científico-acadèmica de gran quantitat de formats i disciplines.\supercite{GoogleScholar}

			      \item \textbf{Microsoft Academic} és un motor de cerca web públic i gratuït per a publicacions acadèmiques, desenvolupat per Microsoft Research.\supercite{MicrosoftAcademic}

			      \item \textbf{ResearchGate} és una xarxa social i una eina de col·laboració per a científics i investigadors per compartir documents, fer i respondre preguntes, i trobar col·laboradors.\supercite{ResearchGate}

			      \item \textbf{arXiv} és un repositori científic per a la publicació d'articles científics en format digital en els camps de les matemàtiques, física, informàtica i biologia que poden ser obtinguts lliurement a través d'Internet.\supercite{ArXiv}

		      \end{itemize}

		\item Gran part de la informació per dur a terme el disseny i la programació dels diferents models de xarxes neuronals en la part pràctica del treball són de fonts lliures. Això vol dir que aquesta informació s'ha obtingut de fòrums, blocs i pàgines web amb tutorials on els autors exposen, expliquen i mostren com fer determinades coses, sense un ànim de lucre. Comparteixen els seus mètodes i les seves conclusions de forma desinteressada. Això implica provar i verificar molts dels mètodes per assegurar el seu funcionament. Per tant, són referències a pàgines web actives i que poden modificar-se i desaparèixer en el temps. Concretament he utilitzat:

		      \begin{itemize}

			      \item \textbf{GitHub} és un servei de hosting de repositoris Git, el qual ofereix tota la funcionalitat de Git de control de revisió distribuït i administració de codi de la font (SCM) així com afegint les seves característiques pròpies.\supercite{GitHub}

			      \item \textbf{Stack Overflow} és una pàgina web de preguntes i respostes per a programadors professionals i entusiastes que forma part de la Xarxa Stack Exchange.\supercite{StackOverflow}

			      \item \textbf{Tutorials Point} és una pàgina web que ofereix una gran quantitat de tutorials, cursos i articles gratuïts relacionats amb temes de llenguatges de programació i altres tecnologies.\supercite{TutorialsPoint}

			      \item \textbf{Medium} és un servei de publicació de blogs.\supercite{Medium}

		      \end{itemize}

	\end{itemize}

	\addtocontents{toc}{\vspace{1em}}
	\printbibliography[heading=subbibintoc]

\end{refsection}