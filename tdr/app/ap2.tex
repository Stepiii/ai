\begin{refsection}
\chapter{Programari utilitzat: \LaTeX}

Tot el treball està escrit utilitzant \LaTeX com a processador de textos. Es la primera vegada que utilitzo aquest programa i ha requerit un esforç inicial extra per aprendre'l. En general, crec que ha valgut la pena fer aquest esforç, ja que la presentació final queda molt millor que si estigués fet amb els programes habituals per processar textos, com ara el Word. A part, aquest esforç es veurà recompensat, ja que en un futur podré utilitzar el \LaTeX per altres treballs, ja que presenta molts avantatges, en especial pel que fa al format de fórmules i la gestió de tot el contingut.

El \LaTeX és programari lliure i és molt utilitzat en l'àmbit acadèmic, especialment a les universitats en els àmbits relacionats amb les ciències (matemàtiques, informàtica, física, ...). Va ser elaborat per L. Lamport a partir del llenguatge TeX inventat per D. Knuth.\supercite{LaTeX}

Dins dels avantatges d’aquest programari vull destacar els següents.

\begin{itemize}
	\item \textit{Format general}. Permet centrar-se exclusivament en el contingut sense preocupar-se en els detalls de format, que es gestionen de manera general i es poden modificar en qualsevol moment de manera rapida.
	
	\item \textit{Estructura general}. Permet estructurar fàcilment el document en capítols, seccions, bibliografia, índex..., amb comptadors i numeració automàtica.
	
	\item \textit{Projecte-arbre}. Es pot crear un fitxer general que crida a altres fitxers, de manera que crea un arbre amb l’estructura dels diferents capítols. Així es poden organitzar els fitxers per capítols, i només et cal obrir el capítol en el qual estàs treballant. Això va bé perquè no cal compilar cada cop tot el treball.
	
	\item \textit{Automatització} de l'índex general i de l'índex de figures.
	
	\item \textit{Fórmules}. Aquest programa ja est`a pensat per tal de poder-hi introduir fórmules matemàtiques. Això és un gran avantatge, ja que es poden escriure tota mena de
	fórmules per ser reproduïdes amb una tipografia professional.
	
	\item \textit{Taules i figures}. Permet crear taules i afileraments amb molts paràmetres possibles. També permet incloure figures de diferents formats. Tant les taules com les figures poden ser definits com a elements flotants per facilitar la seva distribució a les pàgines, la seva enumeració automàtica, el text de peu de figura associat i el llistat final de l'índex de figures.
\end{itemize}

Tot i que com es veu és un programa amb un gran potencial, té també alguns inconvenients. El principal és que cal programar instruccions i, per tant, requereix un temps i un esforç d’aprenentatge. En l’elaboració concreta, les dificultats amb què m’he trobat són les següents:

\begin{itemize}
	\item Tot i que està preparat a escala mundial, he tingut alguns problemes amb els accents.
	\item La inclusió d’imatges no és tan directa com en altres programes, i ha requerit instal·lar algun paquet addicional.
	\item Cal anar compilant per veure com queda, ja que no és WYSIWYG (What You See Is What You Get)
	\item Com que és un programa, la sintaxi ha de ser estrictament correcta, sovint d´ona errors i no es pot compilar. No sempre és fàcil veure l’origen de l’error per poder corregir-lo.
\end{itemize}

\addtocontents{toc}{\vspace{1em}}
\printbibliography[heading=subbibintoc]

\end{refsection}