\begin{refsection}

	\chapter{Introducció}

	\section{Motivació}

	A l'hora de triar el tema del treball de recerca vaig pensar que m'agradaria fer-lo de programació, però no només duna manera abstracta, sinó en relació amb alguna aplicació concreta que tingués a veure amb els meus interessos. Un dia vaig descobrir què eren les xarxes neuronals i la intel·ligència artificial. D'una banda, em vaig adonar que existeixen molts mites i molta desinformació sobre el tema de la IA: la intel·ligència artificial substituirà tots els llocs de treball, superarà la intel·ligència humana o conduirà a l'esclavitud de la nostra raça per part dels éssers robòtics superiors, com es descriu en diverses pel·lícules i històries de ciència-ficció.

	De l'altra banda, em va sorprendre la quantitat d'aplicacions que té la IA actualment  (concretament l'aprenentatge automàtic i les xarxes neuronals) i la seva important presència en el nostre dia a dia. D'aquesta manera, després d'haver fet recerca per informar-me del tema, sem van acudir aquestes preguntes:

	\begin{itemize}
		\item Poden els ordinadors aprendre de l'experiència com nosaltres, humans?
		\item Què són les xarxes neuronals i perquè són tan útils?
		\item Com ho fan les xarxes neuronals per reconèixer imatges?
		\item Quins altres mètodes d'aprenentatge automàtic hi ha a part de les xarxes neuronals?
	\end{itemize}

	A partir d'aquestes preguntes vaig decidir fer un treball relacionat amb la intel·ligència artificial.

	El fet d'haver triat aquest tema també està motivat pel meu interès per les matemàtiques perquè estan relacionades amb la programació. A grans trets el treball consisteix, per una banda, en entendre què és la IA, l'aprenentatge automàtic i les xarxes neuronals, com funcionen, quines són les seves aplicacions, etc. i, per l'altra, en dissenyar i programar models de xarxes neuronals i aplicar-les als entorns reals i virtuals per resoldre problemes concrets.

	\section{Objectius}

	Concretament els objectius d'aquest treball són:

	\begin{itemize}

		\item Investigar en àrees d'IA, informàtica, i programació per aprendre conceptes que poden ser aplicats en diferents àrees com l'anàlisi d'imatges o presa de decisions.

		\item Explicar el funcionament de diferents mètodes d'aprenentatge automàtic (centrant-se principalment en les xarxes neuronals).

		\item Demostrar que els sistemes d'aprenentatge automàtic són útils i aplicables.

		\item Comprendre els conceptes, el funcionament i l'abast de les xarxes neuronals i mostrar el procés de disseny i implementació de xarxes neuronals amb TensorFlow i Keras per aconseguir objectius específics.

	\end{itemize}

	En el camp de visió artificial (Reconeixement d'imatges):

	\begin{itemize}

		\item Dissenyar i implementar un model predictiu eficaç en la classificació de dígits escrits a mà utilitzant el conjunt de dades MNIST.

		\item Analitzar el funcionament i el rendiment de diferents arquitectures de xarxes neuronals.

	\end{itemize}

	\begin{comment}

	Videojocs (presa de decisions):

	\begin{itemize}

		\item Adaptar la metodologia d'aprenentatge per reforçament d'AlphaZero de Google DeepMind per a jocs com connecta 4 o gomoku.

		\item Analitzar el funcionament dels algoritmes emprats en AlphaZero (arbre de cerca de Monte Carlo, MCTS).

		\item Entrenar un agent de Deep Q Learning (DQN) per realitzar tasques de OpenAI Gym (videojocs i simulacions diferents) i analizar el seu rendiment.
	\end{itemize}

	\end{comment}

	\addtocontents{toc}{\vspace{1em}}
	\printbibliography[heading=subbibintoc]

\end{refsection}
