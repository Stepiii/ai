\begin{refsection}

	\chapter{Conclusió}
	\label{chap:conc}

	Com hem pogut observar, les xarxes neuronals proporcionen un gran potencial davant la resolució de problemes de predicció i classificació. En aquest cas, s’ha construït un model capaç de resoldre un problema relativament senzill per un ésser humà, però assolint amb un percentatge d'encert molt elevat.
	
	Tenint en compte que el problema resolt es considera senzill i introductori, la seva resolució il·lumina la idea sobre la diversitat dels problemes que es poden arribar a solucionar mitjançant l'aplicació de les xarxes neuronals en qualsevol context.
	
	Tot i això, les aplicacions que es poden extrapolar del model generat podrien arribar a ser molt útils i amb una bona acceptació comercial.
	
	Pel que fa als objectius fixats al començament del projecte, s'assoleix completament el propòsit principal de dissenyar i programar un model basat en xarxes neuronals capaç de reconèixer
	dígits escrits a mà amb una eficiència elevada.
	
	Aquest treball ha complert definitivament els seus objectius, no només cap a la familiarització amb conceptes de la intel·ligència artificial i aprenentatge automàtic aplicats al processament d'imatges, que és un camp que evoluciona constantment, sinó també s'ha après a dissenyar models de xarxes neuronals per al processament i reconeixement d'imatges.
	
	Afrontar problemes comuns en aquest àmbit com ara triar el \textit{framework} més adequat, triar el mètode d'optimització més eficaç, tractar les dades abans i després del processament, intentar esbrinar el millor mètode d'anàlisi per als resultats, solucionar problemes d'execució derivats de mal funcionament o optimitzar el model, són habilitats útils i de difícil adquisició.
	
	\begin{itemize}
		
		\item Hem aprendre conceptes que poden ser aplicats en diferents àrees com l'anàlisi d'imatges o presa de decisions.
		
		\item Explicar el funcionament de diferents mètodes d'aprenentatge automàtic (centrant-se principalment en les xarxes neuronals).
		
		\item Demostrar que els sistemes d'aprenentatge automàtic són útils i aplicables.
		
		\item Comprendre els conceptes, el funcionament i l'abast de les xarxes neuronals i mostrar el procés de disseny i implementació de xarxes neuronals amb TensorFlow i Keras per aconseguir objectius específics.
		
	\end{itemize}
	
	Reconeixement d'imatges (visió artificial):
	
	\begin{itemize}
		
		\item Dissenyar i implementar un model predictiu eficaç en la classificació de dígits escrits a mà utilitzant el conjunt de dades MNIST.
		
		\item Analitzar el funcionament i el rendiment de diferents arquitectures de xarxes neuronals.
		
	\end{itemize}
	
	\addtocontents{toc}{\vspace{1em}}
	\printbibliography[heading=subbibintoc]

\end{refsection}
