\begin{otherlanguage}{english}

	\chapter*{Abstract}
	\setcounter{chapter}{-4}
	\addcontentsline{toc}{chapter}{Abstract}
	
	In computer science, artificial intelligence (AI) is an intelligence presented by machines, unlike the natural intelligence shown by humans and animals. This field of computing is devoted to the development of algorithms that can be implemented on machines so that they make decisions and, therefore, provide them with a certain intelligence to perform cognitive functions that imitate human beings, such as learning and solving problems. In recent years the field of artificial intelligence has evolved a great deal, to such an extent that today many companies have developed intelligent systems capable of reaching points that no one would have thought before. Most of the AI examples that currently exist, from computers that play chess to voice recognition and autonomous vehicles, largely depend on deep learning and the processing of natural language. By using these technologies, computers can be trained to perform certain tasks by processing large amounts of data and recognizing patterns. In this work, an introduction will be made to the world of neuronal networks and deep learning to try to recognize and classify manuscript digits with a high percentage of success, testing different architectures of neural networks and varying their parameters.
	
\end{otherlanguage}

\newpage

\begin{otherlanguage}{catalan}

	\chapter*{Resum}
	\setcounter{chapter}{-3}
	\addcontentsline{toc}{chapter}{Resum}

	En ciències de la computació, la intel·ligència artificial (IA) és una intel·ligència que presenten les màquines, a diferència de la intel·ligència natural que mostren els humans i els animals. Aquest camp d’informàtica es dedica al desenvolupament d’algoritmes que es puguin implementar en màquines perquè aquestes prenguin decisions i, per tant, dotar-les de certa intel·ligència per realitzar funcions cognitives que imiten a les humanes, com ara \textit{aprendre} i \textit{resoldre problemes}. En els últims anys el camp la intel·ligència artificial ha evolucionat molt, fins a tal punt que avui en dia moltes companyies han desenvolupat sistemes intel·ligents capaços d'arribar a punts que anys enrere ningú s'hauria pensat. La majoria dels exemples de la IA que existeixen actualment, des dels ordinadors que juguen als escacs fins al reconeixement de veu i els vehicles autònoms, depenen en gran manera de l’aprenentatge profund i del processament del llenguatge natural. Utilitzant aquestes tecnologies, els ordinadors poden ser entrenats per realitzar determinades tasques processant grans quantitats de dades i reconeixent patrons. En aquest treball es realitzarà una introducció al món de les xarxes neuronals i l'aprenentatge profund per intentar reconèixer i classificar dígits manuscrits amb un alt percentatge d'encert, provant diferents arquitectures de xarxes neuronals i variant els seus paràmetres.

\end{otherlanguage}

\newpage

\begin{otherlanguage}{spanish}

	\chapter*{Resumen}
	\setcounter{chapter}{-2}
	\addcontentsline{toc}{chapter}{Resumen}

	En ciencias de la computación, la inteligencia artificial (IA) es una inteligencia que presentan las máquinas, a diferencia de la inteligencia natural que muestran los humanos y los animales. Este campo de informática se dedica al desarrollo de algoritmos que se puedan implementar en máquinas para que estas tomen decisiones y, por tanto, dotarlas de cierta inteligencia para realizar funciones cognitivas que imitan a las humanas, tales como aprender y resolver problemas. En los últimos años el campo la inteligencia artificial ha evolucionado mucho, hasta tal punto que hoy en día muchas compañías han desarrollado sistemas inteligentes capaces de llegar a puntos que años atrás nadie habría pensado. La mayoría de los ejemplos de la IA que existen actualmente, desde los ordenadores que juegan al ajedrez hasta el reconocimiento de voz y los vehículos autónomos, dependen en gran medida del aprendizaje profundo y del procesamiento del lenguaje natural. Utilizando estas tecnologías, los ordenadores pueden ser entrenados para realizar determinadas tareas procesando grandes cantidades de datos y reconociendo patrones. En este trabajo se realizará una introducción al mundo de las redes neuronales y el aprendizaje profundo para intentar reconocer y clasificar dígitos manuscritos con un alto porcentaje de acierto, probando diferentes arquitecturas de redes neuronales y variando sus parámetros.

\end{otherlanguage}
