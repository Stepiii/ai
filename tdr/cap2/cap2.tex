\begin{refsection}

	\chapter{Intel·ligència artificial}
	\label{chap:AI}

	\section{Història}
	
	Tot i que els primers referents històrics es remunten als anys 30 amb Alan Turing, considerat el pare de la intel·ligència artificial, es considera que el punt de partida és l'any 1950, precisament, quan Turing publica un article amb el títol \textit{Computing machinery and intelligence} a la revista Mind, on es va plantejar la pregunta: poden pensar les màquines? I proposa un mètode per determinar si una màquina pot pensar. Els fonaments teòrics de la IA es troben en l'experiment que es proposa en aquest article i que va passar a anomenar-se test de Turing. Mitjançant la superació d'aquest test per una màquina, és possible que la màquina pugui passar per un ésser humà. El test manté la seva vigència en l'actualitat i és el motiu d'estudis i investigacions contínues.\supercite{Breuhist}
	
	No obstant això, nombrosos investigadors i historiadors consideren que el punt de partida de la intel·ligència artificial moderna va ser l'any 1956, quan els pares de la intel·ligència artificial moderna, John McCarty, Marvin Misky i Claude Shannon van definir formalment el terme durant la conferència de Darmouth com \textit{la ciència i l'enginy de fer màquines intel·ligents, especialment programes de càlcul intel·ligent}.\supercite{LiveScience}
	
	El supercomputador Deep Blue d'IBM va guanyar el 1997 al campió mundial d'escacs Gari Kaspàrov, després d'un fracàs previ en 1996 on va guanyar Kaspàrov. L'any 1997 és considerat per alguns historiadors de la IA com el punt d'inflexió on va començar a sentir-se de la intel·ligència artificial fora dels àmbits acadèmics i de recerca.\supercite{Breuhist}.
	
	Al febrer de 2011, el supercomputador Watson d'IBM, el model de computador cognitiu, com l'anomena el seu creador IBM, guanya al concurs televisiu dels Estats Units \textit{Jeopardy!}, en el qual es realitzen preguntes i qüestions diferents de tota mena, cultura i coneixement, als dos millors concursants del programa, Brad Ruttler i Ken Jennings.\supercite{aiwiki}
	
	Una altre fet important va ser la presentació d'Apple de l'assistent virtual Siri integrat al telèfon mòbil iPhone 4S a l'any 2011 i on van començar les primeres experiències d'aprenentatge automàtic i els primers indicis d'aprenentatge profund.\supercite{NationalGeographic}
	
	L'any 2012 és considerat com l'any clau de la segona generació d'intel·ligència artificial, amb el llançament d'assistents virtuals recolzats en IA amb algoritmes d'aprenentatge profund. Al juny de 2012 Google va presentar el seu assistent virtual, Google Now, i a l'abril de 2014 Microsoft va presentar el seu propi assistent virtual, Cortana.\supercite{aiwiki}
	
	El 9 de març de 2016, el programari d'intel·ligència artificial Alpha Go de Google DeepMind es va enfrontar al sud-coreà Se-Dol, campió mundial de Go, un joc mil·lenari d'estratègia molt complex, en una partida a cinc jocs. Alpha Go va guanyar els tres primers jocs netament i només en l'últim cinquè joc Se-Dol va guanyar, gràcies a un moviment inicial que va fer i on es va comprovar que la màquina estava poc entrenada per enfrontar-se a situacions inesperades.\supercite{Breuhist}.
	
	\section{Branques de la IA segons les aplicacions}
	En aquesta secció, veurem els diferents camps compatibles amb IA:\supercite{aitutor, aiwiki, edureka}

	\begin{itemize}
		\item \textbf{Aprenentatge automàtic}. L'aprenentatge automàtic és la ciència d'aconseguir que les màquines interpretin, processin i analitzin dades per tal de resoldre problemes del món real.
		
		\item \textbf{Aprenentatge profund}. L’aprenentatge profund és el procés d'implementar les xarxes neuronals en dades d'alta dimensió per obtenir informació i formar solucions. L'aprenentatge profund és un camp avançat d'aprenentatge automàtic que es pot utilitzar per resoldre problemes més avançats.
		
		\item \textbf{Jocs}. La IA té un paper crucial en jocs estratègics com escacs, pòquer, tres en ratlla , etc., on la màquina pot pensar en un gran nombre de posicions possibles basades en coneixements heurístics.
		
		\item \textbf{Processament del llenguatge natural}. És possible interactuar amb l’ordinador que entén el llenguatge natural que parlen els humans.
		
		\item \textbf{Sistemes experts}. Hi ha algunes aplicacions que integren màquina, programari i informació especial per proporcionar raonament i assessorament. Proporcionen explicació i assessorament als usuaris.
		
		\item \textbf{Sistemes de visió}. Aquests sistemes comprenen, interpreten i comprenen l’entrada visual a l’ordinador. Per exemple, un avió d'espionatge fa fotografies, que s’utilitzen per conèixer informació espacial o elaborar mapes de les zones. Els metges utilitzen sistemes experts clínics per diagnosticar els pacients. La policia utilitza programari informàtic que pot reconèixer la cara del criminal amb el retrat elaborat per un artista.
		
		\item \textbf{Reconeixement de veu}. Alguns sistemes intel·ligents són capaços d'escoltar i comprendre el llenguatge en termes d'oracions i els seus significats mentre un humà en parla. Pot manejar diferents accents, paraules d'argot, soroll en segon pla, canvis en la pronunciació a causa del fred, etc.
		
		\item \textbf{Reconeixement d’escriptura}. El programari de reconeixement d'escriptura manual llegeix el text escrit en paper per un bolígraf. Pot reconèixer les formes de les lletres i convertir-la en text editable.
		
		\item \textbf{Robots intel·ligents}. Els robots són capaços de realitzar les tasques donades per un humà. Disposen de sensors per detectar dades físiques del món real com ara llum, calor, temperatura, moviment, so i pressió. Disposen de processadors eficients, múltiples sensors i una enorme memòria per mostrar intel·ligència. A més, són capaços d'aprendre dels seus errors i es poden adaptar al nou entorn.
	\end{itemize}
	

	\addtocontents{toc}{\vspace{1em}}
	\printbibliography[heading=subbibintoc]

\end{refsection}
